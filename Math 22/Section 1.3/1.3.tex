\documentclass{article}
\usepackage{amsmath, amssymb, array, fancyhdr, enumerate, enumitem}
\usepackage[margin = 1in]{geometry}

\def\class{Math 22}
\def\header{1.3 Propositional Equivalences}
\def\first{Jason}
\def\last{Wong}

\fancypagestyle{useheader}
{
  \fancyhf{}
  \setlength{\headsep}{\baselineskip}
  \lhead{\class}
  \chead{\header}
  \rhead{\last, \first}
  \cfoot{\thepage}
}

% logical operators
\let\iff\leftrightarrow
\let\lnot\neg
\let\xor\oplus



% draws a black square for proofs
\newcommand{\qed}{\hfill$\blacksquare$}

\newcommand{\solution}[1]{
  \textbf{Solution} #1
}

\newcommand{\step}[1]{
  \begin{enumerate}
    \item[{}] #1
  \end{enumerate}
}

% args: (equation (aligns at & sign: x &= y), explanation)
\newcommand{\proofstep}[2]{#1 && \text{#2}\\}

% args: (number, question, steps, explanation)
\newenvironment{twocolproof}[4]{
    \begin{enumerate}
        \item[\bfseries{#1}] #2
        \begin{align*}
            #3
        \end{align*}
        #4
    \end{enumerate}
}


% args (number, problems, subproblemss)
\newenvironment{nestedproblem}[3]{
  \begin{enumerate}
    \item[\bfseries{#1}] #2
    #3
  \end{enumerate}
  \hfill
}

% args: (number, question, solution)
\newenvironment{problem}[3]{
  \begin{enumerate}
    \item[\bfseries{#1}] #2
    \begin{enumerate}
      \item[{}] \solution{#3}
    \end{enumerate}
  \end{enumerate}
  \hfill
}

\begin{document}
\pagestyle{useheader}
\begin{nestedproblem}{1} {Use truth tables to verify these equivalences}{
    \problem{a}
    {$p \land T \equiv p$}
    {
      $\begin{array}{c c | c }
        p & T & p \land T\\
        \hline
        T & T & T \\
        F & T & F \\
      \end{array}$
    }
    \problem{b}
    {$p \lor F \equiv p$}
    {
      $\begin{array}{c c | c }
        p & F & p \lor F\\
        \hline
        T & F & T \\
        F & F & F \\
      \end{array}$
    }
    \problem{c}
    {$p \land F \equiv F$}
    {
      $\begin{array}{c c | c }
        p & F & p \land F\\
        \hline
        T & F & F \\
        F & F & F \\
      \end{array}$
    }
    \problem{d}
    {$p \lor T \equiv T$}
    {
      $\begin{array}{c c | c }
        p & F & p \lor T\\
        \hline
        T & T & T \\
        F & T & T \\
      \end{array}$
    }
    \problem{e}
    {$p \lor p \equiv p$}
    {
      $\begin{array}{c | c }
        p & p \lor p\\
        \hline
        T & T \\
        F & F \\
      \end{array}$
    }
    \problem{f}
    {$p \land p \equiv p$}
    {
      $\begin{array}{c | c }
        p & p \land p\\
        \hline
        T & T \\
        F & F \\
      \end{array}$
    }
  }
\end{nestedproblem}

\problem{6}{Use a truth table to verify the first De Morgan law\\
  $\lnot (p \land q) \equiv \lnot p \lor \lnot q$
}
{
  $\begin{array}{c c | c | c | c | c | c}
    p & q & p \land q & \lnot(p \land q) & \lnot p & \lnot q & \lnot p \lor \lnot q\\
    \hline
    T & T & T & F & F & F & F \\
    T & F & F & T & F & T & T \\
    F & T & F & T & T & F & T \\
    F & F & F & T & T & T & T \\
  \end{array}$
}
\pagebreak
\begin{nestedproblem}{7} {Use De Morgan’s laws to find the negation of each of the
  following statements.}{
  \problem{a}
  {Jan is rich and happy.}
  {
    \\
    Let p = Jan is rich\\
    Let q = Jan is happy\\

    $p \land q = $ Jan is rich and happy\\
    $\lnot(p \land q) = \lnot p \lor \lnot q = $ Jan is not rich or happy
  }
  \problem{b}
  {Carlos will bicycle or run tomorrow.}
  {
    \\
    Let p = Carlos will bicycle tomorrow\\
    Let q = Carlos will run tomorrow\\

    $p \lor q = $ Carlos will bicycle or run tomorrow\\
    $\lnot(p \lor q) = \lnot p \land \lnot q = $ Carlos will not bicycle and run tomorrow
  }
  \problem{c}
  {Mei walks or takes the bus to class}
  {
    \\
    Let p = Mei walks to class\\
    Let q = Mei takes the bus to class\\

    $p \lor q = $ Mei walks or takes the bus to class\\
    $\lnot(p \lor q) = \lnot p \land \lnot q = $ Mei does not walk and take the bus to class
  }
  \problem{d}
  {Ibrahim is smart and hard working.}
  {
    \\
    Let p = Ibrahim is smart\\
    Let q = Ibrahim is hard working\\

    $p \land q = $ Ibrahim is smart and hard working\\
    $\lnot(p \land q) = \lnot p \lor \lnot q = $ Ibrahim is not smart or hard working
  }
}
\end{nestedproblem}
\pagebreak
\begin{nestedproblem}{9} {Show that each of these conditional statements is a tautology by using truth tables.}{
  \problem{a}
  {$(p \land q) \to p$}
  {
    $\begin{array}{c c | c | c }
      p & q & p \land q & (p \land q) \to p\\
      \hline
      T & T & T & T \\
      T & F & F & T \\
      F & T & F & T \\
      F & F & F & T \\
    \end{array}$
  }
  \problem{b}
  {$p \to (p \lor q)$}
  {
    $\begin{array}{c c | c | c }
      p & q & p \lor q & p \to (p \lor q)\\
      \hline
      T & T & T & T\\
      T & F & T & T\\
      F & T & T & T\\
      F & F & F & T\\
    \end{array}$
  }
  \problem{c}
  {$\lnot p \to (p \to q)$}
  {
    $\begin{array}{c c | c | c | c}
      p & q & \lnot p & p \to q & \lnot p \to (p \to q)\\
      \hline
      T & T & F & T & T \\
      T & F & F & F & T \\
      F & T & T & T & T \\
      F & F & T & T & T \\
    \end{array}$
  }
  \problem{d}
  {$(p \land q) \to (p \to q)$}
  {
    $\begin{array}{c c | c | c | c}
      p & q & p \land q & p \to q & (p \land q) \to (p \to q)\\
      \hline
      T & T & T & T & T \\
      T & F & F & F & T \\
      F & T & F & T & T \\
      F & F & F & T & T \\
    \end{array}$
  }
  \problem{e}
  {$\lnot (p \to q) \to p$}
  {
    $\begin{array}{c c | c | c | c}
      p & q & p \to q & \lnot (p \to q) & \lnot (p \to q) \to p\\
      \hline
      T & T & T & F & T \\
      T & F & F & T & T \\
      F & T & T & F & T \\
      F & F & T & F & T \\
    \end{array}$
  }
  \problem{f}
  {$\lnot (p \to q) \to \lnot q$}
  {
    $\begin{array}{c c | c | c | c}
      p & q & p \to q & \lnot (p \to q) & \lnot (p \to q) \to q\\
      \hline
      T & T & T & F & T \\
      T & F & F & T & T \\
      F & T & T & F & T \\
      F & F & T & F & T \\
    \end{array}$
  }
}
\end{nestedproblem}

\pagebreak
\begin{nestedproblem}{11} {Show that each of these conditional statements is a tautology by using truth tables.}{
  \begin{twocolproof}{a}
  {$(p \land q) \to p$}
  {
    \proofstep{(p \land q) &\to p}{Given}
    \proofstep{\lnot(p \land q) &\lor p}{Table 7}
    \proofstep{\lnot p \lor \lnot q &\lor p}{De Morgan's law}
    \proofstep{\lnot p \lor p &\lor \lnot q}{Commutative law}
    \proofstep{T &\lor \lnot q}{Idempotent law}
    \proofstep{&T}{Domination law}
  }{}
  \end{twocolproof}
  \begin{twocolproof}{b}
  {$p \to (p \lor q)$}
  {
    \proofstep{p &\to (p \lor q)}{Given}
    \proofstep{\lnot p &\lor (p \lor q)}{Table 7}
    \proofstep{(\lnot p &\lor p) \lor q}{Associative Law}
    \proofstep{T &\lor q}{Negation Law}
    \proofstep{&T}{Domination Law}
  }{}
  \end{twocolproof}
  \begin{twocolproof}{c}
  {$\lnot p \to (p \to q)$}
  {
    \proofstep{\lnot p &\to (p \to q)}{Given}
    \proofstep{p &\lor (\lnot p \lor q)}{Table 7}
    \proofstep{(p &\lor \lnot p) \lor q}{Associative Law}
    \proofstep{T &\lor q}{Negation Law}
    \proofstep{&T}{Domination Law}
  }{}
  \end{twocolproof}
  \begin{twocolproof}{d}
  {$(p \land q) \to (p \to q)$}
  {
    \proofstep{(p \land q) &\to (p \to q)}{Given}
    \proofstep{\lnot (p \land q) &\lor (p \to q)}{Table 7}
    \proofstep{\lnot p \lor \lnot q &\lor (p \to q)}{De Morgan's law}
    \proofstep{\lnot p \lor \lnot q &\lor (\lnot p \lor q)}{Table 7}
    \proofstep{\lnot p \lor \lnot p &\lor \lnot q \lor q}{Commutative Law}
    \proofstep{\lnot (p \land p) &\lor \lnot q \lor q}{De Morgan's Law}
    \proofstep{\lnot p &\lor \lnot q \lor q}{Idempotent law}
    \proofstep{\lnot p &\lor T}{Negation law}
    \proofstep{&T}{Domination law}
  }{}
  \end{twocolproof}
  \begin{twocolproof}{e}
  {$\lnot (p \to q) \to p$}
  {
    \proofstep{\lnot (p \to q) &\to p}{Given}
    \proofstep{\lnot (\lnot p \lor q) &\to p}{Table 7}
    \proofstep{p \land \lnot q &\to p}{De Morgan’s law}
    \proofstep{\lnot(p \land \lnot q) &\lor p}{Table 7}
    \proofstep{\lnot p \lor q &\lor p}{De Morgan’s law}
    \proofstep{\lnot p \lor p &\lor q}{Commutative law}
    \proofstep{T &\lor q}{Negation law}
    \proofstep{&T}{Domination law}
  }{}
  \end{twocolproof}
  \begin{twocolproof}{f}
  {$\lnot (p \to q) \to \lnot q$}
  {
    \proofstep{\lnot (p \to q) &\to \lnot q}{Given}
    \proofstep{\lnot (\lnot p \lor q) &\to \lnot q}{Table 7}
    \proofstep{p \land \lnot q &\to \lnot q}{De Morgan’s law}
    \proofstep{\lnot(p \land \lnot q) &\lor \lnot q}{Table 7}
    \proofstep{\lnot p \lor q &\lor \lnot q}{De Morgan’s law}
    \proofstep{\lnot p &\lor T}{Negation law}
    \proofstep{&T}{Domination law}
  }{}
  \end{twocolproof}
}
\end{nestedproblem}

\begin{twocolproof}{14}
  {Determine whether $(\lnot p \land (p \to q)) \to \lnot q$ is a tautology}
  {
    \proofstep{(\lnot p \land (p \to q)) &\to \lnot q}{Given}
    \proofstep{(\lnot p \land (\lnot p \lor q)) &\to \lnot q}{Table 7}
    \proofstep{\lnot (\lnot p \land (\lnot p \lor q)) &\lor \lnot q}{Table 7}
    \proofstep{\lnot ((\lnot p \land \lnot p) \lor q) &\lor \lnot q}{Associative law}
    \proofstep{\lnot (\lnot p \lor q) &\lor \lnot q}{Idempotent law}
    \proofstep{p \land \lnot q &\lor \lnot q}{De Morgan’s law}
    \proofstep{p \land (\lnot q &\lor \lnot q)}{Associative law}
    \proofstep{p &\land \lnot q}{Idempotent law}
  }{
    $(\lnot p \land (p \to q)) \to \lnot q$ is the logical equivalent of $p \land \lnot q$, which is not a tautology and is therefore is not a tautology
  }
\end{twocolproof}
\problem{20}
{Show that $\lnot(p \xor q)$ and $p \iff q$ are logically equivalent}
{
  $\begin{array}{c c | c | c | c}
    p & q & p \xor q & \lnot(p \xor q) & p \iff q\\
    \hline
    T & T & F & T & T \\
    T & F & T & F & F \\
    F & T & T & F & F \\
    F & F & F & T & T \\
  \end{array}$
}
\pagebreak
\problem{27}
{Show that $p \iff q$ and $(p \to q) \land (q \to p)$ are logically equivalent}
{
  $\begin{array}{c c | c | c | c | c}
    p & q & p \to q & q \to p & (p \to q) \land (q \to p) & p \iff q\\
    \hline
    T & T & T & T & T & T \\
    T & F & F & T & F & F \\
    F & T & T & F & F & F \\
    F & F & T & T & T & T \\
  \end{array}$
}

\begin{twocolproof}{32}
  {Show that $(p \land q) \to r$ and $(p \to r) \land (q \to r)$ are not logically equivalent}
  {
    \proofstep{(p \to r) &\land (q \to r)}{Given}
    \proofstep{(p \to r) \land (q \to r) &\equiv (p \lor q) \to r}{Table 7}
    \proofstep{(p \land q) \to r &\not\equiv (p \lor q) \to r}{}
  }{}
\end{twocolproof}

\problem{41}
{Find a compound proposition involving the propositional
variables p, q, and r that is true when exactly two of p, q,
and r are true and is false otherwise.}
{
  $(p \land q \land \neg r) \lor (p \land r \land \neg q) \lor (q \land r \land \neg p)$
}

\problem{58}
{How many of the disjunctions $p \lor \lnot q$, $\lnot p \lor q$, $q \lor r$, $q \lor \lnot r$, $\lnot q \lor \lnot r$ can be made simultaneously true by an assignment of truth values to p,q, and r?}
{
  Let p, q, r = T\\
  $p \lor \lnot q = T$\\
  $\lnot p \lor q = T$\\
  $q \lor r = T$\\
  $q \lor \lnot r = T$\\
  $\lnot q \lor \lnot r = F$\\
}
\problem{58}
{How many of the disjunctions $p \lor \lnot q\lor s$, $\lnot p\lor \lnot r\lor s$, $\lnot p\lor \lnot r\lor \lnot s$, $\lnot p\lor q\lor \lnot s$, $q\lor r\lor \lnot s$, $q\lor \lnot r\lor \lnot s$, $\lnot p\lor \lnot q\lor \lnot s$, $p\lor r\lor s$, and $p\lor r\lor \lnot s$ can  be  made  simultaneously  true  by  an  assignment  of truth values to p, q, r, and s?}
{
  Let p, q, r, s = T\\
  $p \lor \lnot q \lor s = T$\\
  $\lnot p\lor \lnot r\lor s = T$\\
  $\lnot p\lor \lnot r\lor \lnot s = F$\\
  $\lnot p\lor q\lor \lnot s = $\\
  $q\lor r\lor \lnot s = T$\\
  $q\lor \lnot r\lor \lnot s = T$\\
  $\lnot p\lor \lnot q\lor \lnot s = F$\\
  $p\lor r\lor s = T$\\
  $p\lor r\lor \lnot s = T$\\
}
\pagebreak
\begin{nestedproblem}{61} {Determine whether each of these compound propositions is satisfiable.}{
  \problem{a}
  {$(p\lor \lnot q) \land (\lnot p\lor q) \land (\lnot p\lor \lnot q)$}
  {
    \\
    $p\lor \lnot q$ can be satisfied when either p = T or q = F\\
    $\lnot p\lor q$ can be satisfied when either p = F or q = T\\
    $\lnot p\lor \lnot q$ can be satisfied when either p = F or q = F\\
    This proposition can be satisfied if both p and q are false.
  }
  \problem{b}
  {$(p\to q)\land (p\to \lnot q)\land (\lnot p\to q)\land (\lnot p\to \lnot q)$}
  {
    $\begin{array}{c c | c | c | c | c | c | c}
      p & q & \lnot p & \lnot q & p\to q & p\to \lnot q & \lnot p\to q & \lnot p\to \lnot q\\
      \hline
      T & T & F & F & T & F & T & T \\
      T & F & F & T & F & T & T & F \\
      \hline
      F & T & T & F & T & T & T & T \\
      \hline
      F & F & T & T & T & T & F & T \\
    \end{array}$\\
    This proposition can be satisfied if p = false and q = true
  }
  \problem{c}
  {$(p \iff q) \land (\lnot p \iff q)$}
  {
    This proposition cannot be satisfied because p cannot be true if and only if q while being simultaneously false if and only if q
  }
}
\end{nestedproblem}
\begin{nestedproblem}{61} {Determine whether each of these compound propositions is satisfiable.}{
  \problem{a}
  {$(p \lor q \lor \lnot r)\land (p \lor \lnot q \lor \lnot s)\land (p \lor \lnot r \lor \lnot s)\land (\lnot p \lor \lnot q \lor \lnot s)\land (p \lor q \lor \lnot s)$}
  {
    All propositions here are disjunctions and contain at least one negation so it can be satisfied if p, q, r, and s are false
  }
  \problem{b}
  {$(\lnot p \lor \lnot q \lor r)\land (\lnot p \lor q \lor \lnot s)\land (p \lor \lnot q \lor \lnot s)\land (\lnot p \lor \lnot r \lor \lnot s)\land (p \lor q \lor \lnot r)\land (p \lor \lnot r \lor \lnot s)$}
  {
    All propositions here are disjunctions and contain at least one negation so it can be satisfied if p, q, r, and s are false
  }
  \problem{c}
  {$(p\lor q\lor r) \land (p\lor \lnot q\lor \lnot s) \land (q\lor \lnot r\lor s) \land (\lnot p\lor r\lor s) \land (\lnot p\lor q\lor \lnot s) \land (p\lor \lnot q\lor \lnot r) \land (\lnot p\lor \lnot q\lor s) \land (\lnot p\lor \lnot r\lor \lnot s)$}
  {
    All propositions here contain at least 1 non negated truth value except for $(\lnot p\lor \lnot r\lor \lnot s)$, which requires either p r or s to be false. No other proposition depends on r being true so setting it to false satisfies this proposition.
  }
}
\end{nestedproblem}

\end{document}