\documentclass{article}
\usepackage{amsmath, amssymb, amsthm, array, fancyhdr, enumerate, enumitem, stackengine}
\usepackage[margin = 1in]{geometry}

\def\class{Math 22}
\def\header{1.7 Introduction to Proofs}
\def\first{Jason}
\def\last{Wong}

\fancypagestyle{useheader}
{
  \fancyhf{}
  \setlength{\headsep}{\baselineskip}
  \lhead{\class}
  \chead{\header}
  \rhead{\last, \first}
  \cfoot{\thepage}
}

% logical operators
\let\iff\leftrightarrow
\let\lnot\neg
\let\xor\oplus



% draws a black square for proofs
\renewcommand{\qed}{\hfill$\blacksquare$}

\newcommand{\solution}[1]{
  \textbf{Solution} #1
}

\newcommand{\step}[1]{
  \begin{enumerate}
    \item[{}] #1
  \end{enumerate}
}

% args: (equation (aligns at & sign: x &= y), explanation)
\newcommand{\proofstep}[2]{&#1 &&\ \  \text{#2}}

% args: (number, question, steps, explanation)
\newenvironment{twocolproof}[1]{
    \begin{align}
      #1
    \end{align}
    \setcounter{equation}{0}
}
% \newenvironment{twocolproof}[4]{
%     \begin{enumerate}
%         \item[\bfseries{#1}] #2
%         \begin{align}
%             #3
%         \end{align}
%         #4
%     \end{enumerate}
%     \setcounter{equation}{0}
% }


% args (number, problems, subproblemss)
\newenvironment{nestedproblem}[3]{
  \begin{enumerate}
    \item[\bfseries{#1}] #2
    #3
  \end{enumerate}
  \hfill
}

% args: (number, question, solution)
\newenvironment{problem}[3]{
  \begin{enumerate}
    \item[\bfseries{#1}] #2
    \begin{enumerate}
      \item[{}] #3
    \end{enumerate}
  \end{enumerate}
  \hfill
}

\begin{document}
\pagestyle{useheader}
  \begin{problem}{1}{Use a direct proof to show that the sum of two odd integers is even.}{
    \begin{proof}
      Assume $a$ is even and $b$ is odd.
      $$\text{Let }a = 2k, \text{ k is some integer}$$
      $$\text{Let }b = 2k + 1, \text{ k is some integer}$$
      \begin{twocolproof} {
        \proofstep{b + b = n}{The sum of 2 odd integers}\\
        \proofstep{2k + 1 + 2k + 1 = n}{Expand b}\\
        \proofstep{4k+2 = n}{Simplify}\\
        \proofstep{2(2k+1) = n}{Factor out 2}\\
        \proofstep{2b = n}{Definition of an even number, b is some integer}
      }
      \end{twocolproof}
    \end{proof}
  }
  \end{problem}
  \begin{problem}{3}{Show that the square of an even number is an even number using a direct proof.}{
    \begin{proof}
      Assume $a$ is even.
      $$\text{Let }a = 2k, \text{ k is some integer}$$
      \begin{twocolproof} {
        \proofstep{a * a = n}{The square of two even numbers}\\
        \proofstep{2k * 2k = n}{Expand a}\\
        \proofstep{2 * 2k^2 = n}{Rearrange terms}\\
        \proofstep{\text{Let } j = 2k^2}{}\\
        \proofstep{2j = n}{Definition of an even number, j is some integer}
      }
      \end{twocolproof}
    \end{proof}
  }
  \end{problem}
  \begin{problem}{4}{Show that the additive inverse, or negative, of an even number is an even number using a direct proof.}{
    \begin{proof}
      Assume $a$ is even.
      $$\text{Let }a = 2k, \text{ k is some integer}$$
      \begin{twocolproof} {
        \proofstep{a * -1 = n}{Additive inverse of an even number}\\
        \proofstep{2k * -1 = n}{Expand}\\
        \proofstep{2(-k) = n}{Rearrange terms}\\
        \proofstep{\text{Let } j = -k}{}\\
        \proofstep{2j = n}{Definition of an even number, j is some integer}
      }
      \end{twocolproof}
    \end{proof}
  }
  \end{problem}
  \begin{problem}{6}{Use a direct proof to show that the product of two odd numbers is odd.}{
    \begin{proof}
      Assume $a$ is odd.
      $$\text{Let }a = 2k + 1, \text{ k is some integer}$$
      $$\text{Let i be some integer}$$
      $$\text{Let j be some integer}$$
      \begin{twocolproof} {
        \proofstep{ai * aj = n}{Product of 2 odd numbers}\\
        \proofstep{i(2k + 1) * j(2k + 1) = n}{Expand}\\
        \proofstep{(2k + 1)\left(i * j*(2k + 1)\right) = n}{Rearrange terms}\\
        \proofstep{\text{Let } m = \left(i * j*(2k + 1)\right)}{}\\
        \proofstep{am = n}{Definition of an odd number, m is some integer}
      }
      \end{twocolproof}
    \end{proof}
  }
  \end{problem}
  \begin{problem}{8}{Prove that if $n$ is a perfect square, then $n+2$ is not a perfect square.}{
    \begin{proof}
      We can define $a$ as a perfect square
      $$\text{Let }a = k^2, \text{ k is some integer}\\$$
      We can prove this by contradiction by assuming that if $n = i^2$ then $n+2 = j^2$. 
      \begin{twocolproof} {
        \proofstep{n+2 = j^2}{Assumption}\\
        \proofstep{2 = j^2 - n}{Subtract n from both sides}\\
        \proofstep{2 = j^2 - i^2}{Replace n with $i^2$}\\
        \proofstep{2 = (j - i)(j + i)}{Factor}\\
        \proofstep{
          \frac{\stackon{+j + i = 2}{j - i = 1}}{2j = 3}
          }{}\\
        \proofstep{j = \frac{2}{3}}{j is not an integer so this is a contradiction}
      }
      \end{twocolproof}
    \end{proof}
  }
  \end{problem}
  \pagebreak
  \begin{problem}{9}{Use a proof by contradiction to prove that the sum of an irrational number and a rational number is irrational.}{
    \begin{proof}
      We can define a rational number as a fraction of 2 integers.
      $$\text{Let }r = \frac{p}{q}, \text{ p and q are integers, $q \not = 0$}\\$$
      Let x be an irrational number and let y be a rational number. To prove this by contradiction we will assume that the sum of x and y are rational. Using the definition of a rational number we can say
      $$x + y = \frac{p}{q}, \text{ p and q are integers, $q \not = 0$}$$
      and that
      $$y = \frac{a}{b}, \text{ a and b are integers, $q \not = 0$}$$
      such that
      $$x + \frac{a}{b} = \frac{p}{q}$$
      From here we can subtract $\frac{a}{b}$ from both sides
      $$x = \frac{p}{q} - \frac{a}{b} = \frac{pb - aq}{qb}$$
      a, b, p, and q are all nonzero integers. This would imply that x is rational, which contradicts our original assumption that x is irrational. Therefore our assumption that x + y is rational is false.
      $$\text{\\}$$
    \end{proof}
  }
  \end{problem}
  \begin{problem}{13}{Prove that if x is irrational, then 1/x is irrational.}{
    \begin{proof}
      We can define a rational number as a fraction of 2 integers.
      $$\text{Let }r = \frac{p}{q}, \text{ p and q are integers, $q \not = 0$}\\$$
      Let x be an irrational number. To prove this by contradiction we will assume that 1/x is rational. Using the definition of a rational number we can say
      $$\frac{1}{x} = \frac{p}{q}, \text{ p and q are integers, $q \not = 0$}$$
      Taking the reciprocal of both sides we get
      $$x = \frac{q}{p}$$
      This would imply that x is a rational number, which contradicts our assumption that 1/x is rational if x is irrational. Therefore our assumption that $\frac{1}{x}$ is rational is false
      $$\text{}$$
    \end{proof}
  }
  \end{problem}
  \pagebreak
  \begin{problem}{17}{Show that if $n$ is an integer and $n^3+5$ is odd, then $n$ is even}{
    \begin{problem}{a}{Using proof by contraposition}{
      \begin{proof}
        We can define an even number as $a$ and an odd number as $b$ where
        $$\text{Let }a = 2k, \text{ k is some integer}$$
        $$\text{Let }b = 2k + 1, \text{ k is some integer}$$
        To prove this by contraposition, we must show that that if $n$ is \textbf{not} even then $n^3+5$ is \textbf{not} odd. In other words, if $n$ is odd then $n^3+5$ is even. We will let n be an odd number such that
        $$n = 2k + 1, \text{ k is some integer}$$
        so we can write this statement as
        $$m = (2k + 1)^3 + 5$$
        Expanding this out gives us
        $$m = 8k^3 + 12k^2 + 6k + 6$$
        From here, we can factor a 2 from the entire polynomial
        $$m = 2(4k^3 + 6k^2 + 3k + 3)$$
        Since k is an integer, the result of $4k^3 + 6k^2 + 3k + 3$ will also be an integer, which we will define as c. We can then say
        $$m = 2c$$
        Which matches the definition of an even number. Thus we have proved that that if $n$ is odd then $n^3+5$ is even
        $$$$
      \end{proof}
    }
    \end{problem}
    \pagebreak
    \begin{problem}{b}{Using proof by contradiction}{
      \begin{proof}
        We can define an even number as $a$ and an odd number as $b$ where
        $$\text{Let }a = 2k, \text{ k is some integer}$$
        $$\text{Let }b = 2k + 1, \text{ k is some integer}$$
        To prove this by contradiction we will assume that if $n^3+5$ is odd, then $n$ is odd. We will let n be an odd number such that
        $$n = 2k + 1, \text{ k is some integer}$$
        and that the result of $(2k + 1)^3 + 5$ is odd such that 
        so we can write this statement as
        $$2j + 1 = (2k + 1)^3 + 5, \text{ j is some integer}$$
        Expanding this out gives us
        $$2j + 1 = 8k^3 + 12k^2 + 6k + 6$$
        From here, we can factor a 2 from the right side
        $$2j + 1 = 2(4k^3 + 6k^2 + 3k + 3)$$
        Since k is an integer, the result of $4k^3 + 6k^2 + 3k + 3$ will also be an integer, which we will define as c. We can then say
        $$2j + 1 = 2c$$
        This is saying that an even number is equal to an odd number, which is a contradiction. Therefore if $n^3+5$ is odd, then $n$ must be even
        $$$$
      \end{proof}
    }
    \end{problem}
  }
  \end{problem}
  \begin{problem}{19}{Prove the proposition $P(0)$, where $P(n)$ is the proposition "If n is a positive integer greater than 1, then $n^2 > n$." What kind of proof did you use?}{
    \begin{proof}
      The proposition states that if $n > 1$ then $n^2 > n$ so for $P(0)$, if $0 > 1$ then $0 > 0$. This is trivially true by vacuous proof.
      $$\text{}$$
    \end{proof}
  }
  \end{problem}
  \begin{problem}{23}{Show that at least ten of any 64 days chosen must fall on the same day of the week.}{
    \begin{proof}
      We will prove this using a contradiction. Assume that only 9 or less of any 64 days fall on the same day of the week. This means we can choose $9 * 7 = 63$ days, which is less than 64 days, so this is a contradiction.
      $$\text{}$$
    \end{proof}
  }
  \end{problem}
  \pagebreak
  \begin{problem}{27}{Prove that if n is a positive integer, then $n$ is odd if and only if $5n+6$ is odd}{
    \begin{proof}
      Because this is a biconditional statement, our proof will have 2 parts. First we will prove that if $n$ is odd then $5n+6$ is odd using a direct proof. To do this we will need to define an odd number $a$ and an even number $b$ as such
      $$\text{Let }a = 2k + 1, \text{ k is some integer}$$
      $$\text{Let }b = 2k, \text{ k is some integer}$$
      We will let n be some odd number such that
      $$n = 2k + 1$$
      and that
      $$5(2k + 1) + 6 = 2j + 1, \text{ j is some integer}$$
      Distributing the 5 to $2k + 1$ gives us 
      $$10k + 5 + 6 = 2j + 1$$
      Which can then be rewritten as
      $$10k + 10 + 1 = 2j + 1$$
      We can now factor out a 2 from $10k + 10$
      $$2(5k + 5) + 1 = 2j + 1$$
      k is an integer so the result of $5k + 5$ will also be an integer which we will represent with $c$
      $$2c + 1 = 2j + 1$$
      Using the definition of an odd number, we have shown that if n is odd then $5n+6$ is odd.\\\\
      We now need to prove that if $5n+6$ is odd then $n$ is odd. This time we will use a proof by contradiction. Suppose if $5n+6$ is odd, then $n$ is not odd. Using the definitions of even and odd numbers, we will let n be some even number such that
      $$n = 2k, \text{ k is some integer}$$
      and we will let $5n+6$ be an odd number such that
      $$5(2k)+6=2j + 1, \text{ j is some integer}$$
      Multiplying out the left side gives us
      $$10k+6=2j + 1$$
      We can factor out a 2 from $10k + 6$
      $$2(5k+3)=2j + 1$$
      And because k is an integer, the result of $5k+3$ will also be an integer which we will represent with $c$
      $$2c=2j + 1$$
      Using the definition of even and odd numbers, this contradicts our supposition that if $5n+6$ is odd, then $n$ is not odd.
      Therefore, we have shown that if $5n+6$ is odd then $n$ is odd.
      $$\text{}$$
    \end{proof}
  }
  \end{problem}
  \pagebreak
  \begin{problem}{35}{Are these steps for finding the solutions of $\sqrt{x+3} = 3-x$ correct?
    \begin{twocolproof} {
      \proofstep{\sqrt{x+3} = 3-x}{Given}\\
      \proofstep{x+3=x^2-6x+9}{Square both sides of (1)}\\
      \proofstep{0=x^2-7x+6}{Subtract x+3 from both sides of (2)}\\
      \proofstep{0=(x-1)(x-6)}{Factor the right-hand side of (3)}\\
      \proofstep{x=1\text{ or }x=6,}{Follows from (4) because $ab=0$ implies that $a=0$ or $b=0$}
    }
    \end{twocolproof}
    }{
    This is not correct because $3-6$ would yield a negative number which is not in the range of a square root
  }
  \end{problem}
\end{document}